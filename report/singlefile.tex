\documentclass[a4paper; 11pt]{article}

%%%%%%%%%%%%%%%%%%%%%%%%%%%%%%%%%%%%%%%%%%%%%%%%%%%%%%%%%%%%%%%%%%%%%%%%%%%%%%%
% includes
%%%%%%%%%%%%%%%%%%%%%%%%%%%%%%%%%%%%%%%%%%%%%%%%%%%%%%%%%%%%%%%%%%%%%%%%%%%%%%%
\usepackage[polish]{babel}	% polski
\usepackage[utf8]{inputenc}
\usepackage{polski}		    % polski
%%% fix for conflict polish babel and amssymb
\makeatletter
\let\polishletl=\lll \let\polishletL=\LLL
\let\lll\relax \let\LLL\relax % undefine them
\def\plll{\polishletl}
\def\pLLL{\polishletL}
\makeatother

\usepackage{courier} 		% times, kurier
\usepackage{amsmath}		% mathematic
\usepackage{amssymb}		% mathematic - symbols
\usepackage{amsfonts}		% mathematic - fonts
\usepackage{graphicx}		% for graphics/pictures
\usepackage{subcaption}     % subfigures (many figures as one)
\usepackage{wrapfig}        % figures wraped in text
\usepackage{geometry}		% for changing page layout
\usepackage{indentfirst}	% indent in every paragraph
\usepackage{icomma}		    % inteligent commas
\usepackage{booktabs}		% enchanced tables
\usepackage{float}		    % for pictures/graphics placed where you want
\usepackage[locale=FR]{siunitx} % pretty units
%\sisetup{per-mode=fraction}
%\sisetup{per-mode=reciprocal}
\sisetup{per-mode=symbol}
\usepackage{verbatim}		% simple block-comments
\usepackage{color}          % easy text colouring
\usepackage{courier} 		% times, courier
\usepackage{hyperref}       % magical clickable references
\usepackage{csquotes}		% better quotations \enquote

\usepackage[
% set style, examples: 
% https://www.overleaf.com/learn/latex/Biblatex_citation_styles
style=alphabetic,
sorting=none, language=autobib, autolang=other, backref=true, isbn=true, 
url=false, maxbibnames=3, backend=biber
]{biblatex}
\addbibresource{references.bib}

%\sisetup{per-mode=fraction}
%\sisetup{per-mode=reciprocal}
\sisetup{per-mode=symbol}
%\newgeometry{tmargin=2.3cm, lmargin=1.9cm, rmargin= 1.9cm, bmargin= 2.3cm}

%%%%%%%%%%%%%%%%%%%%%%%%%%%%%%%%%%%%%%%%%%%%%%%%%%%%%%%%%%%%%%%%%%%%%%%%%%%%%%%
% local includes
%%%%%%%%%%%%%%%%%%%%%%%%%%%%%%%%%%%%%%%%%%%%%%%%%%%%%%%%%%%%%%%%%%%%%%%%%%%%%%%

\renewcommand{\figurename}{Rys.}
\renewcommand{\tablename}{Tab.}
\renewcommand{\abstractname}{Abstrakt}
%\newcommand{\mc}[3]{\multicolumn{#1}{#2}{#3}}

\title{Tytuł}
\author{Paweł Rzońca}
\date{\today}

\begin{document}

\maketitle

%%%%%%%%%%%%%%%%%%%%%%%%%%%%%%%%%%%%%%%%%%%%%%%%%%%%%%%%%%%%%%%%%%%%%%%%%%%%%%%
% content here
%%%%%%%%%%%%%%%%%%%%%%%%%%%%%%%%%%%%%%%%%%%%%%%%%%%%%%%%%%%%%%%%%%%%%%%%%%%%%%%
Lorem ipsum dolor sit amet, consectetur adipiscing elit. 
Proin nibh augue, suscipit a, scelerisque sed, lacinia in, mi. 
Cras vel lorem. Etiam pellentesque aliquet tellus. Phasellus pharetra 
nulla ac diam. Quisque semper justo at risus. 
Donec venenatis, turpis vel hendrerit interdum, dui ligula ultricies purus, 
sed posuere libero dui id orci. Nam congue, pede vitae dapibus aliquet, 
elit magna vulputate arcu, vel tempus metus leo non est. 
Etiam sit amet lectus quis est congue mollis. 
Phasellus congue lacus eget neque~\cite{x}.
Phasellus ornare, ante vitae consectetuer consequat, purus sapien 
ultricies dolor, et mollis pede metus eget nisi. 
Praesent sodales velit quis augue. Cras suscipit, urna at aliquam rhoncus, 
urna quam viverra nisi, in interdum massa nibh nec erat.


\printbibliography

\end{document}
